\renewcommand{\kHz}{\si{\kHz}\xspace}
\renewcommand{\MHz}{\si{\MHz}\xspace}
\renewcommand{\W}{\si{\W}\xspace}
\renewcommand{\s}{\si{\s}\xspace}
\newcommand{\Oppers}{~\si[per-mode=symbol]{Op\per\s}\xspace}


\section{Results}

\subsection{Synthesis}

In order to evaluate the performance of our improved processor we need first to evaluate the
performance of the baseline version of the processor.

The synthesis tool reported the values shown in table~\ref{tbl:resource_utilization} for the resources utilization.

Notice that the value ``P/f'' indicates the energetic efficiency of the processor. Another useful insight is the fact that the power
consumption is almost proportional to the clock frequency, therfore we can use this value to estimate the
power consumption at different clock frequencies.

From the synthesis report we can also see the slowest path which determines the max clock
frequency.

This path is from \texttt{``ddrsp0.ddrc0/ddr32.ddrc/ra.raddr\_0''} to \texttt{``ddrsp0. ddrc0/ddr\_phy0/ddr\_phy0/xc4v.ddr\_phy0/casgen''}.

Those components belong to the SDRAM controller and the path is located between the controller
and the physical interface with the memory.

The synthesis of our modified version gave us the results showed as well in table~\ref{tbl:resource_utilization}:

\begin{table}[H]
\centering
\begin{tabular}{cccc>{\color{white}\columncolor{Cyan}}c}
\bottomrule
 & Baseline & Only Multiplier & Only Divider & Final Version \\
Max clk freq. [\MHz] & 80.522 & - & 80.535 & 40.197 \\
\# of Occupied Slices & 9904 & -  & 10479 & 11886\\
Total \# of 4-input LUTs & 16889 & -  & 17865 & 20469\\
Quiescent power [\W] & 2.467 & - & 2.468 & 2.511\\
Dynamic power [\W] & 0.721 & - & 0.743 & 0.832\\
Total power [\W] & 3.188 & - & 3.211 & 3.343\\
P/f [\si[per-mode=symbol]{\W\per\MHz}] & 0.03959 & - & 0.03987 & 0.08317\\\toprule
\end{tabular}
\caption{Resource Utilization}
\label{tbl:resource_utilization}
\end{table}

As expected the area consumption is moderately higher than before as the units we designed are more complex than the baseline. In particular the lookup table in the divider, and the Wallace tree structure in the multiplier are very space-hungry.

Also the power consumption increased for the same reason, the algorithms are more complicated so
more energy is expended for all the calculations. However this trade-off was envisioned at the start of the project, and as we will see later the benchmark's performance improved.

The most significant change is in the maximum clock frequency which was drastically reduced. The main culprit is the multiplier which only takes one clock cycle to finish. Originally, the unit was designed to complete after two cycles. However it was verified that when the processor is configured with a two-cycle latency, it expects the 32x32 operation to finish in one cycle. Thus the multiplier had to be adjusted for this behavior.

Alternatively, the \texttt{mul32} unit could be adapted to the remaining multiplier configuration (\emph{i.e.}~4-cycle latency) so the maximum frequency would not be so severely affected.

\subsection{Benchmark Scores}

Once the processor is synthesized and has been loaded in a FPGA board, Linux is initiated on the processor along with several benchmarks. In table~\ref{tbl:benchmarks_baseline} the execution times of these
benchmarks are reported, for detailed scores please see the accompanying Excel file.

\begin{table}[H]
\centering
\begin{tabular}{c>{\color{white}\columncolor{Cyan}}c}
\bottomrule

Stanford [\s] & 2.30\\
Whetstone [\s] & 116.2\\
Gmpbench Multiply [\Oppers] & 781\\
Gmpbench Divide [\Oppers] & 15876\\
Gmpbench RSA [\Oppers] & 5123\\
Division [\s] & 8.06\\
Mibench JPEG (average) [\s] & 23.215\\
SSD [\s] & 10.59\\
Total [\s] & \textbf{219.28}\\
\toprule
\end{tabular}
\caption{Benchmarks Scores Baseline.}
\label{tbl:benchmarks_baseline}
\end{table}

The scores obtained with the modified processor are reported in table~\ref{tbl:benchmarks_modified}.

\begin{table}[H]
\centering
\begin{tabular}{c>{\color{white}\columncolor{Cyan}}c}
\bottomrule

Stanford [\s] & 2.21\\
Whetstone [\s] & 112.08\\
Gmpbench Multiply [\Oppers] & 914\\
Gmpbench Divide [\Oppers] & 19205\\
Gmpbench RSA [\Oppers] & 5353\\
Division [\s] & 7.31\\
Mibench JPEG (average) [\s] & 21.76\\
SSD [\s] & 8.60\\
Total [\s] & \textbf{206.92}\\
\toprule
\end{tabular}
\caption{Benchmarks Scores Modified Version.}
\label{tbl:benchmarks_modified}
\end{table}


From these results we can see that almost every benchmark had a healthy improvement, in the end
the total execution time improved by 5.6\%.

These scores are not as good as expected but likely the execution of the operating system
on the processor causes a non negligible overhead in the execution due to the scheduling. As an example, the
divider takes about half the time to execute but the execution time of the division related
benchmarks is only about 10\% better.

\subsection{Metrics Comparison}

In order to get a fair comparison between the baseline and the improved processor some standard
metrics have to be calculated and studied.

Usually these metrics are \texttt{A}, the area consumption here calculated as the weighted sum of the
number of occupied slices and the number of 4-input LUTs used where the weight is the reciprocal
of the number of available resources, \texttt{D} is the delay or the reciprocal of the max clock frequency
and indicate the delay of the slowest path, \texttt{P} is the power and it's simply calculated as the total
power consumed by the Dhrystone benchmark used for the simulation and \texttt{BS} is the benchmark
score which indicate how fast a program can be executed, it's calculated as the total execution
time of the benchmarks on the FPGA board.

For the area we are using that complex formula because in order to have a parameter which indicate the true cost of our implementation we needed something more than just the number of resources used.
We assumed that the number of resource available on the device is inverse proportional to their effective cost, so if we use their reciprocal as a weight to calculate the average, we obtain a metric which is most likely nearer to the effective cost of the implementation once implemeted on chip.

Moreover some composite metrics can be observed: these metrics consider two or more primitive
metrics and often are more interesting than the latter because any improvement in one metric is usually accompanied by decrease in other. Composite metrics show the overall performance, and give insight on the trade-offs.

Since we want to speed-up the execution of the software while keeping the power consumption
low because this is a processor designed for embedded applications, the composite metric we are
interested the most is \texttt{P$\times$BS}. It reflects how much the processor is able to compute for the same amount of energy.

Additional composite metrics are \texttt{A$\times$D}, \texttt{A$\times$BS} and \texttt{P$\times$D}. Since we focused on the
execution time and power consumption one can notice that the other metrics are worse in our version
compared to the baseline.

All the baseline's and modified units' synthesis and benchmark results have been condensed in table~\ref{tbl:final_metrics}.

 Due to the important impact the multiplier has on these metrics, results are shown before and after this custom unit is integrated.

\begin{table}[H]
\centering
\begin{tabular}{ccccc}

&
\multicolumn{4}{c}{Primitive Metrics}\\
Version &
\ttfamily A &
\ttfamily D &
\ttfamily P &
\ttfamily BS \\
Baseline &
\num{2.68E+04} &
\num{1.24E-02} &
\num{3.19E+00} &
\num{2.19E+02} \\
Modified (Div)&
\num{2.83E+04} &
\num{1.24E-02} &
\num{3.21E+00} &
\num{2.13E+02} \\
Modified (Mul\&Div)&
\num{3,24E+04} &
\num{2,49E-02} &
\num{3,34E+00} &
\num{2,07E+02} \\
Improvements (Div) &
\color{red} 5.8s\% &
- &
\color{red} 0.7\% &
\color{green} -2.7\% \\
Improvements (Mul\&Div) &
\color{red} 21\% &
\color{red} 100\% &
\color{red} 5\% &
\color{green} -6\% \\

\midrule

& \multicolumn{4}{c}{Composite Metrics}\\
Version &
\ttfamily A$\times$D &
\ttfamily A$\times$BS &
\ttfamily P$\times$D &
\ttfamily P$\times$BS\\
Baseline &
\num{3.33E+02} &
\num{5.88E+06} &
\num{3.96E-02} &
\num{6.99E+02} \\
Modified (Div)&
\num{3.52E+02} &
\num{6.05E+06} &
\num{3.99E-02} &
\num{6.85E+02} \\
Modified (Mul\&Div)&
\num{8,05E+02} &
\num{6,69E+06} &
\num{8,32E-02} &
\num{6,92E+02} \\
Improvements (Div) &
\color{red} 5.8s\% &
\color{red} 2.9\% &
\color{red} 0.7\% &
\color{green} -2.0\% \\
Improvements (Mul\&Div)&
\color{red} 142\% &
\color{red} 14\% &
\color{red} 110\% &
\color{green} -1\%\\
\end{tabular}
\caption{Final Metrics For Baseline And Improved Versions.}
\label{tbl:final_metrics}
\end{table}