\pagenumbering{arabic}

\section{Introduction}

For the Processor Design Project course we have been asked to improve the performance of the
LEON3, a 32-bit SPARC V8 processor designed for embedded applications.
Our main target is to decrease the computation time for certain benchmarks keeping the power
consumption as low as possible, thus for us the most relevant compound metric is the
\texttt{power}$\times$\texttt{benchmark score} (\texttt{P}$\times$\texttt{BS}).

The SPARC V8 architecture contemplates the use of instruction and special hardware for integer
multiplications and divisions, but with the original configuration the multiplier takes 5 clock cycles
to calculate the result and the divider 36, so one of the first things we decided to do is to improve
these arithmetic cores. Simple algorithms can be implemented to obtain significant improvements.

The division is not a very common operation, even if in the baseline version is quite slow and can be improved, but the multiplication on the other hand is very common and apart from the calculation for the application software, can be also used to calculate addresses for array access, and so almost every kind of benchmark we are going to run on the processor, as well as the operative system, can take benefit by an improvement, this if of course the compiler is smart enough to use the dedicated istruction when needed.

In section 2 an analysis of the current baseline of the processor is done in order to find the weak points that will be improved.
Then in section 3 the improvement of the arithmetic unit, the multiplier and divider, is discussed.
In section 4 the results from the simulation, synthesis and the FPGA board are presented and discussed, it's also discussed our methodology to compare different designs based on some metrics.
In the end some further improvements of the processor are suggested for future works.

\pagebreak
\section{Baseline Analysis and Working Plan}

\pagebreak
\section{Improved Arithmetic Cores}

In order to improve the performance of the arithmetic unit we redesigned from the ground up both the
two multiplier and divider units.

In order to make them compatible with the rest of the processor we studied in a detailed way all
the handshaking signals.

The original multiplier can be configured for ``2 cycles'' of latency (instead of 5) through the use of \texttt{make xconfig}.
It is important to note that for this particular setup the ready signal is not used.
So although we are writing a new multiplier it is necessary to configure the
processor with a 2-cycle multiplier as well, so the processor can handle the
handshaking signals generated by our multiplier in the correct way.

For the divider there is not a previous configuration, so the processor knows that an operation is
completed by inspecting the \texttt{ready} and \texttt{nready} signals which have been reproduced following the
specifications.

The part of the core that handles the other signals such as \texttt{start}, \texttt{flush} or \texttt{holdn}, has been designed
to mimic the original version, thereafter all the handshaking signals are handled and generated following the
specifications to enable unit compatibility with the processor.