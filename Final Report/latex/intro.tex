\pagenumbering{arabic}

\section{Introduction}

For the Processor Design Project course we have been asked to improve the performance of the
LEON3, a 32-bit SPARC V8 processor designed for embedded application.
Our main target is to decrease the computation time for certain benchmarks keeping the power
consumption as low as possible, so the main compound metric we are going to care of is the
\texttt{power}$\times$\texttt{benchmark} score (\texttt{P}$\times$\texttt{BS}).

The SPARC V8 architecture contemplates the use of instruction and special hardware for integer
multiplications and divisions, but with the original configuration the multiplier takes 5 clock cycles
to calculate the result and the divider 36, so one of the first things we decided to do is to improve
these arithmetic cores, some easy algorithms can be implemented to have a real improvement.

Some other changes are described at the end of this document.

\section{Improved Arithmetic Cores}

In order to improve the performance of the arithmetic unit we designed from the beginning the
two multiplier and divider units.

In order to make them compatible with the rest of the processor we studied in a detailed way all
the handshaking signals.

For the multiplier we noticed that in the baseline version with the configuration ``2 cycles'' the
ready signals are not used, so even if we designed a new multiplier we had to configure the
processor with a 2-cycle multiplier as well, by doing this the rest of the processor can handle the
handshaking signals generated by our multiplier.

For the divider there is not a previous configuration, so the processor knows that an operation is
completed looking at the \texttt{ready} and \texttt{nready} signals which have been reproduced following the
specifications.

The part of the core that handle the other signals such as \texttt{start}, \texttt{flush} or \texttt{holdn}, has been designed
like the original version, so all the handshaking signals are handled and generated following the
specifications and so make the units compatible with the processor.