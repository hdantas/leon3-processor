\pagenumbering{arabic}

\section{Introduction}

For the Processor Design Project course we have been asked to improve the performance of the
LEON3, a 32-bit SPARC V8 processor designed for embedded applications.
Our main target is to decrease the computation time for certain benchmarks keeping the power
consumption as low as possible, thus for us the most relevant compound metric is the
\texttt{power}$\times$\texttt{benchmark score} (\texttt{P}$\times$\texttt{BS}).

The SPARC V8 architecture contemplates the use of instruction and special hardware for integer
multiplications and divisions, but with the original configuration the multiplier takes 5 clock cycles
to calculate the result and the divider 36, so one of the first things we decided to do is to improve
these arithmetic cores. Simple algorithms can be implemented to obtain significant improvements.

The division is not a very common operation, even if in the baseline version is quite slow and can be improved, on the other hand the multiplication is quite frequent. Apart from the calculations for the application software, the multiplier may also be used to calculate addresses for array access, and therefore the benchmarks we will run on the processor, as well as the operating system, can benefit greatly from the improvements.

In section~\ref{sec:baseline} an analysis of the current baseline of the processor is done in order to find the weak points that will be improved.
Thereafter in section~\ref{sec:arth} the improvements to the arithmetic unit, \emph{i.e.} the multiplier and divider, are discussed.
In section~\ref{sec:results} the results from the simulation, synthesis and the FPGA board are presented and discussed, the section also examines our methodology to compare different designs based on various metrics.
In the end some further improvements of the processor are suggested for future work.