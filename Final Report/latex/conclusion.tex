\section{Conclusions and further improvements}

The improvements made to the arithmetic units have improved the benchmark performance of the processor. Although they are modest, they reflect the chosen target scenario.

Because of lack of time we were unable to perform more changes, but of course there are many things to
change in the architecture to improve further the performance.

Different configurations for the multiplier should have been tested to decrease its impact on power and maximum frequency.

The size and structure of the cache memory could be changed to decrease the probability of misses and consequently the benchmarks execution time. Although this could have been done using the configuration tool it
would have not been fruit of our own merit. Moreover increasing the cache size probably would
have increased also the power consumption, escalating even more this issue.

The LEON3 uses a static branch prediction in the integer unit, which is a good compromise
between power consumption, as no difficult computation is needed, yet there are gains in terms of execution time.
To improve performance further, a 1 bit or a 2 bit branch prediction buffer
algorithm could be implemented. The calculations needed are more intricate and computed more often (30\% of
the instructions are branches) henceforth the power consumption would probably increase. On the other hand the gains in terms of execution time could make up for the loss.

In the end another heavy modification that could have been done is making the
integer unit super-scalar and implementing Out of Order execution. Hypothetically it could yield 
significant performance improvements in terms of execution time. Notwithstanding a complete re-design of the integer unit
would have been needed, which deemed it impossible to complete in the (small) time budget for this project.