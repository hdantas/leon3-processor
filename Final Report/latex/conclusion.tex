\section{Conclusions and further improvements}

The improvements have not been as good as we expected but now our modified processor is more
preformat from an energetic point of view making it more suitable for embedded applications.

Because of lack of time we didn't do any other changes, but of course there are many things to
change in the architecture to improve further the performance.

The size and structure of the cache memory can be changed to decrease the probability of misses
and so the benchmarks execution time, but this can be done using the configuration tool and so it
would have not been an our real achievement, moreover increasing the cache size probably would
have increased also the power consumption making things worse.

The LEON3 uses a static branch prediction in the integer unit, which is a good compromise
between power consumption, because no difficult computation is needed, and gain in terms of
execution time. To improve performance further a 1 bit or a 2 bit branch prediction buffer
algorithm. The calculation needed is more complicated and it has to be done very often (30% of
the instructions are branches) so the power consumption probably would increase, but the gain in
terms of execution time could be worth it.

In the end another heavy modification that could have been done could have been making the
integer unit super-scalar and implementing Out of Order execution. This could have make us gain
significant performance in terms of execution time, but a complete re-design of the integer unit
would have been needed, and it would have been impossible to do in such a short time.