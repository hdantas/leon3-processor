\section{Conclusions and further improvements}
\label{sec:conclusion}

The main goal of this project was to improve the Leon 3 processor running on a Xilinx Virtex4 FPGA. Given these abstract requirements it was decided to focus on the power-performance trade off, reflected in the composite metric \texttt{Power $\times$ Benchmark Scores}. To accomplish that we dedicated the bulk of our time to improve the multiplier and divider arithmetic units, as we considered it to be the most efficient way (considering our goal and the necessary time) to distribute the available human resources. At the end we obtained an improvement of 1.1\% for the aforementioned metric. We were expecting stronger gains given the significant decrease in the latency of the units, but this may be due to the overhead of the FPGA's Operating System's scheduler. Notwithstanding, we deem our work successful as it accomplished the prime objective.

Nonetheless there is room for improvement in order to obtain more impressive results, most importantly to diminish the impact of the multiplier in the circtuit frenquecy, but also deeper changes in the architecture could be considered. However, due to lack of time we were unable to perform more changes other than rewriting the multiplier and divider.
As an example, different configurations for the multiplier could have been tested to understand if it would be feasible to decrease its impact on frequency, power and area while keeping or increasing the benchmark scores.

Likewise, the size and structure of the cache memory could have been adjusted to decrease the probability of misses and consequently the benchmarks' execution time. Although this could have been done using the configuration tool it would have not been fruit of our own merit. Moreover increasing the cache size probably would have augmented also the power consumption.

The LEON3 uses a static branch prediction in the integer unit, which is a good compromise
between power consumption, as no difficult computation is needed, yet there are gains in terms of execution time.
To improve performance further, a 1 bit or a 2 bit branch prediction buffer
algorithm could be implemented. The calculations needed are more intricate and computed more often (30\% of
the instructions are branches) henceforth the power consumption would probably increase. On the other hand the gains in terms of execution time could make up for the loss.

In the end another heavy modification that could have been done was making the
integer unit super-scalar and implementing Out of Order execution. Hypothetically it could yield 
significant performance improvements in terms of execution time. Notwithstanding a complete re-design of the integer unit
would have been needed, which deemed it impossible to complete in the (small) time budget for this project.


On a different perspective, in our humble opinion, there are some improvements that can be done on the organization of the Processor Design Course, and in particular this project that would have been helpful for the duration of the project.

The Leon 3 is a very complex process with many moving parts. Perhaps, it would be more educative to have a simpler baseline project, easier to fully comprehend and to debug (also quicker to synthesize). Moreover, with such a project there would likely be a more direct relation between changes in the arithmetic (or other) units and the benchmark scores.

In addition, the reliability of the FPGA server could be improved. In particular at the beginning of the course there were some issues that complicated debugging of our designs. Related to testing it would be \emph{excellent} if we would have remote access to debugging data from the FPGA. Since whenever the benchmarks fail we have no information of what went wrong.

